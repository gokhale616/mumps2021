%% fig 1 
\newcommand{\capInc}{
\textbf{Mumps in the United States.} (A) Total reported mumps cases per $10^5$; (B) Age-stratified case reports across 5 age cohorts (Purple gradient). (C) Age-distribution of mumps case reports (Purple gradient); Spatial distribution of average mumps case reports per $10^5$ during reemergence eras - (D) 1985-1989 and (E) 2006-2012.
}

%% fig 2 
\newcommand{\capFitplot}{
\textbf{Model-data agreement.} (A) Age-specific qualitative model fits to within sample time-series (1977-2012, seagreen) and the corresponding 5 year out of sample prediction (2013-2018, orange) for the waning hypothesis. Ribbons represent 80\% prediction intervals for the two prediction epochs. (B) Age-specific log-linear model agreement for within sample fit (seagreen lines and hollow circles) and out of sample predictions (orange linesand hollow circles). Statistical performance of the model was assessed using coeffcient of variation ($R^2$, reported inset) for the two predction epochs. Year of record is represented using a continuous grey gradient.   
}

\newcommand{\capKLdiv}{
\textbf{Relative age distribution of mumps true incidence.} (A) Qualitative comparsion between expected, assuming the model of waning immunity, and the observed incidence per $10^5$ age-distribution (purple gradient, left, y-axis). Simulated incidence is broken down into the two prediction epochs - within sample (seagreen gradient) and out of sample (orange gradient) to match the analysis presented in figure~\ref{fig:fit_plot}. Lines represent comparsion of temporal shifts in the mean age of first infection (right, y-axis) under the waning model (dotted) and observed incidence (dashed).  (B) Quantitative agreement between observed \textit{v/s.} expected age-distribution of incidence using the Kullback-Leibler Divergence (KLD, left y-axis) through time. Boxplots represent bootsrapped distribution of KLD calculated by comparing observed age-distribution to 1000 synthetically generated time-series under the estimated observation noise. The area plot represents an estimate of age-aggregated mumps incidence per $10^5$ (right, y-axis). 
}

\newcommand{\capSumplot}{
\textbf{Dissecting dynamics of mumps remergence.} (A) Rate of vaccine immunity decay under an exponentially distributed waning model ($\frac{1}{\delta} = $111.5 years) post vaccination. Green dashed line represents the crtical vaccnation level calculated using $R_0 = 14$ for the waning model. (B) Weekly changes in percent effectively vaccinated across age cohorts (red gradient), and total (yellow). (C) Weekly changes in percent susceptible across age cohorts. (D) Weekly dynamics in the effective reproductive number highlighting supercritical (red) and subcritical (green) epidemic signatures (left, y-axis). Annual vaccination rates of neonatal (light green) and booster (dark blue) doses (right, y-axis). (D) Expected weekly loss of vaccine derived immunity per $10^5$. (E) Prevalence per $10^5$ of mumps across various age classes.}

\newcommand{\capVaccEffplot}{
\textbf{Immune duration simulation study.} Relative age-specific (A) risk ($\frac{\lambda_i^w(t)}{\lambda_i^s(t)}$), and (B) prevalence ($\frac{I_i^w(t)}{I_i^s(t)}$) of getting infected in the years after vaccination across the five age-cohorts (facet columns) as a reponse to varying duration of vaccine protection (colour gradient), and at the estimated MLE (indigo); (C) Stable-state prevalence distribution across five age-cohorts for varying duration of immune protection. Covariate values were fixed at the last known value in the year 2018. Dynamics were simulated for 100y and final values in the infectious comprtments values we taken to be prevalence per $10^5$}


%% fig 3
\newcommand{\capImpCoverage}{
\textbf{Reconstructed neonatal vaccine rate.} (A) Time series for the available mumps annual vaccine coverage rates for the neonatal dose. The blue region highlights the period of missing vaccine data. (B) Weekly mumps incidence during the period where vaccine data is missing. (C) Imputed, weekly vaccine uptake (upper panel) by recursively filtering from 1000 stochastically generated trajectories conditioned on the initial decline of mumps in (B). Filtering distribution is assumed to be an over-dispersed normal ($\psi = 0.8$, $\rho = 0.04$). Averaged annual coverage (lower panel) is used in the testing of hypothesis.
}


%% fig 5
\newcommand{\capCov}{
\textbf{Model covariates.} (A-F) Demographic and epidmeiological variables used in introducing realizm to the simulated dynamics; (A) Probability that for a given case age was reported; (B) Total normalized annual births; Annual mumps vaccination rates for the neonatal dose (C) and booster dose (D), Missing early uptake values were interpolated (blue); (E) Age-specific population size; And (F) migration rates broken down by age cohorts.
}

%% fig 4
\newcommand{\capContact}{
\textbf{Matrix of daily contact rates.} Daily diary records of who (y-axis) comes in contact with whom (x-axis) for the UK in 2005. Contact rates were calibarated for the US using age-specific national population sizes and then corrected for reciprocity.
}


\newcommand{\capRelFitplot}{
\textbf{Relative Model Fit.}  Qualitative agreement of age-specific observed cases of mumps (facet rows, black) to 4 models of vaccine imperfection (facet columns) - No Loss (violet), Waning (Exponential, seagreen), Waning (Erlang, N = 3, darkblue), and Leaky (pink).
Colored solid lines represent median trajectories where as ribbons repsresent 80\% prediction intervals. Relative quantitative model-data agreement has been conveyed using Akaike information criterion ($\Delta$AIC, reported inset) among various models.
}


\newcommand{\capAuxParamplot}{
\textbf{Age-specific parameter estimates of Process and Observation model.} Faceted barplots represent maximum likelihood estimates of probability of infection (q), reporting probability ($\rho$) and dispersion parameter ($\psi$) corresponding to the four hypotheses of mumps reemergence (columns). Age-class has been reported using a blue gradient.
}

