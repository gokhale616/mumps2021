%% fig 1 
\newcommand{\capInc}{
\textbf{Mumps incidence in the United States.} (A) Cases of mumps per 100,000 of national population size; (B) Root raw cases of mumps across 5 age cohorts (Purple gradient). (C) Contribution of age cohorts (colours) to mumps case volume; (D-E) State-wise distribution of mumps incidence during resurgence eras.
}

%% fig 2 
\newcommand{\capFitplot}{
\textbf{Model-data agreement.} (A) Qualitative model fits within sample time-series (1977-2012, seagreen) and corresponding out of sample prediction (2013-2018, orange) for the waning hypothesis. Ribbons represent 80\% Predictiona intervals for within-sample fits (seagreen) and outs of sample prediction (orange). (B) Quantitatve measure of agreement was assessed using coefficeint of variation, ($R^2$, reported inset) estimated at a log scale, for within sample era (seagreen) and out of sample era (orange). 
}

\newcommand{\capSumplot}{
\textbf{Dissecting dynamics of mumps remergence.} (A) Represents the rate of vaccine dervied immune decay under an exponentially distributed waning model ($\frac{1}{\delta} = $111.5 years) since the time of vaccination. (B) Weekly percent effectively vaccinated, total (yellow), and across various age cohorts (orange gradient). Green dashed line represents the crtical vaccnation level calculated using $R_0 = 14$ for the waning model. (C) Individuals per $10^5$ who weekly lose their vaccine derived immunity across various cohorts. (D) Weekly effective reproductive number through time that highlights regions of super-critical (red) and subcritical (sea-green) epidemic transitions.  (E) Prevalence per $10^5$ of mumps across various age classes model. (F) Comparison between expected (organge gradient) and observed (purple gradient) age distribution of incidence per $10$
}



%% fig 3
\newcommand{\capImpCoverage}{
\textbf{Reconstructed neonatal vaccine rate.} (A) Time series for the available mumps annual vaccine coverage rates for the neonatal dose. The blue region highlights the period where vaccine data is missing. (B) Weekly mumps incidence over the  period where vaccine data is missing. (C) Weekly vaccine uptake trajectories, upper  panel, were imputed using a particle filter using over-dispersed normal ($\psi = 0.8$, $\rho = 0.04$) weigths. Trajectories were averaged to calculate the annual vaccine cover, lower panel, to be used in the model fotting analysis. 1000 particles were used in the filtering procedure.   
}


%% fig 5
\newcommand{\capCov}{
\textbf{Model covariates.} (A-F) Demographic and epidmeiological variables were used to introduce realizm to the simulated dynamics; (A) Probability that for a given case age was reported; (B) Total normalized annual births; Annual mumps vaccination rates for the neonatal dose (C) and booster dose (D), Missing early uptake values were interpolated (sky blue); (E) Age-specific population size and (F) migration rates broken down by age cohorts.
}

%% fig 4
\newcommand{\capContact}{
\textbf{Matrix of daily contact rates.} Daily diary records of who (y-axis) comes in contact with whom (x-axis) for the UK in 2005. Contact rates were calibarated for the US using age-specific national population sizes and then corrected for reciprocity.
}


\newcommand{\capKLdiv}{
\textbf{Kullback-Leibler Divergence.} 
}

\newcommand{\capRelFitplot}{
\textbf{Relative Model Fit.} 
}


\newcommand{\capAuxParamplot}{
\textbf{Auxillary parameter estimates of Process and Observation model.} Faceted barplots represent age specfic maximum likelihood estimates (blue gradient) of probability of infection (q), reporting probability ($\rho$) and dispersion parameter ($\psi$) corresponding to the the hypotheses tested.
}

