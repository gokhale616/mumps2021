%% fig 1 
\newcommand{\capInc}{
\textbf{Mumps incidence in the United States.} (A) Cases of mumps per 100,000 of national population size; (B) Root raw cases of mumps across 5 age cohorts (colors). (C) Contribution of age cohorts (colours) to mumps case volume; (D-E) State-wise distribution of mumps incidence during resurgence eras.
}

\newcommand{\capFitplot}{
\textbf{Model Data agreement.} (A) Qualitative model fits for two eras, within sample (1977-2012, seagreen) and corresponding out of sample prediction (2013-2018, orange) for the waning hypothesis. (B) Quantitatve measure of agreement was assessed using coefficeint of variation, ($R^2$, reported inset) estimated at a log scale, for within sample era (seagreen) and out of sample era (orange). 
}

\newcommand{\capImpCoverage}{
\textbf{Reconstructed neonatal vaccine rate.} Weekly vaccine uptake trajectories, lower panel, were reconstrtucted using a particle filter with over-dispersed normal ($\psi = 0.8$, $\rho = 0.04$) weigths. Trajectories were averaged to calculate the annual vaccine cover, upper panel. 1000 particles were used in the filtering process.   
}

%% fig 2
\newcommand{\capContact}{
\textbf{Matrix of daily contact rates.} Daily diary records of who (y-axis) comes in contact with whom (x-axis) for the UK in 2005. Contact rates were calibarated for the US using age-specific national population sizes and then corrected for reciprocity.
}

%% fig 3
\newcommand{\capCov}{
\textbf{Model covariates.} (A-F) Demographic and epidmeiological variables were used to introduce realizm to the simulated dynamics; (A) Probability that for a given case age was reported; (B) Total normalized annual births; Annual mumps vaccination rates for the neonatal dose (C) and booster dose (D), Missing early uptake values were interpolated (sky blue); (E) Age-specific population size and (F) migration rates broken down by age cohorts.
}


