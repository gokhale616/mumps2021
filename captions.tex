%% fig 1 
\newcommand{\capInc}{
\textbf{Mumps incidence in the United States.} (A) Cases of mumps per 100,000 of national population size; (B) Root raw cases of mumps across 5 age cohorts (Purple gradient). (C) Contribution of age cohorts (colours) to mumps case volume; (D-E) State-wise distribution of mumps incidence during resurgence eras.
}

%% fig 2 
\newcommand{\capFitplot}{
\textbf{Model-data agreement.} (A) Qualitative model fits within sample time-series (1977-2012, seagreen) and corresponding out of sample prediction (2013-2018, orange) for the waning hypothesis. Ribbons represent 80\% Prediction intervals corresponding to the two prediction epochs. (B) Quantitatve measure of agreement was assessed using coefficeint of variation, ($R^2$, reported inset) estimated assessing the log-linear variation explained by the model corresponding to the two prediction epochs. 
}

\newcommand{\capKLdiv}{
\textbf{Comparing age distribution of observed \textit{v/s.}expected true incidence of mumps.} (A) Qualitative agreement of age-distribution of incidence per $10^5$, as expected under the waning model, to the observed age-distribution (purple gradient). Simulated incidence has been broken down into two prediction epochs - within sample (sea-green gradient) and out of sample (orange gradient to match the analysis presented in figure~\ref{fig:fit_plot}. (B) Quantitative agreement between observed \textit{v/s.} expected age-distribution of incidence using the Kullback-Leibler Divergence (KLD, left y-axis) through time. Boxplots represent bootsrapped distribution of KLD calculated by comparing observed age-distribution to 1000 synthetically generated time-series assuming observeation noise. The area plot represents an estimate of age-aggregated mumps incidence per $10^5$ (right y-axis). 
}

\newcommand{\capSumplot}{
\textbf{Dissecting dynamics of mumps remergence.} (A) Rate of vaccine immunity decay in a cohort under an exponentially distributed waning model ($\frac{1}{\delta} = $111.5 years) post vaccination. (B) Weekly percent effectively vaccinated, Total (yellow), and across various age cohorts (red gradient). Green dashed line represents the crtical vaccnation level calculated using $R_0 = 14$ for the waning model. (C) Weekly loss of vaccine derived immunity per $10^5$. (D) Weekly dynamics in the effective reproductive number highlighting supercritical (red) and subcritical (green) epidemic signatures.  (E) Prevalence per $10^5$ of mumps across various age classes.
}



%% fig 3
\newcommand{\capImpCoverage}{
\textbf{Reconstructed neonatal vaccine rate.} (A) Time series for the available mumps annual vaccine coverage rates for the neonatal dose. The blue region highlights the period where vaccine data is missing. (B) Weekly mumps incidence over the  period where vaccine data is missing. (C) Weekly vaccine uptake trajectories, upper  panel, were imputed using a particle filter using over-dispersed normal ($\psi = 0.8$, $\rho = 0.04$) weigths. Trajectories were averaged to calculate the annual vaccine cover, lower panel, to be used in the model fotting analysis. 1000 particles were used in the filtering procedure.   
}


%% fig 5
\newcommand{\capCov}{
\textbf{Model covariates.} (A-F) Demographic and epidmeiological variables were used to introduce realizm to the simulated dynamics; (A) Probability that for a given case age was reported; (B) Total normalized annual births; Annual mumps vaccination rates for the neonatal dose (C) and booster dose (D), Missing early uptake values were interpolated (sky blue); (E) Age-specific population size and (F) migration rates broken down by age cohorts.
}

%% fig 4
\newcommand{\capContact}{
\textbf{Matrix of daily contact rates.} Daily diary records of who (y-axis) comes in contact with whom (x-axis) for the UK in 2005. Contact rates were calibarated for the US using age-specific national population sizes and then corrected for reciprocity.
}


\newcommand{\capRelFitplot}{
\textbf{Relative Model Fit.}  Qualitative agreement of age-specific observed cases of mumps (facet rows, black) to 4 models of vaccine imperfection (facet columns) - No Loss (violet), Waning (Exponential, seagreen), Waning (Erlang, N = 3, darkblue), and Leaky (pink).
Colored solid lines represent median trajectories where as ribbons repsresent 80\% prediction intervals. Relative quantitative model-data agreement has been conveyed using Akaike information criterion ($\Delta$AIC, reported inset) among various models.
}


\newcommand{\capAuxParamplot}{
\textbf{Age-specific parameter estimates of Process and Observation model.} Faceted barplots represent maximum likelihood estimates of probability of infection (q), reporting probability ($\rho$) and dispersion parameter ($\psi$) corresponding to the four hypotheses of mumps reemergence (columns). Age-class has been reported using a blue gradient.
}

